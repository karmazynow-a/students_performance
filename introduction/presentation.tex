%----------------------------------------------------------------------------
%  PREAMBUŁA
%----------------------------------------------------------------------------

\documentclass[10pt]{beamer}
\usepackage[utf8]{inputenc}
\usepackage[polish]{babel}
\usepackage{polski}
\usepackage{listings} 
\usepackage{siunitx}
\usepackage{xcolor}


            
\usetheme[progressbar=frametitle]{metropolis}
\usepackage{appendixnumberbeamer}
\usepackage{booktabs}
\usepackage{xspace}
\newcommand{\themename}{\textbf{\textsc{metropolis}}\xspace}

\title{Sztuczne sieci neuronowe - założenia projektu}
\date{20 czerwca 2020}
\author{Aleksandra Poręba \and Grzegorz Podsiadło }
\institute{Wydział Fizyki i Informatyki Stosowanej \\ul. Reymonta 19 \\30-055 Kraków \\ Polska}
%\titlegraphic{\hfill\includegraphics[height=1.5cm]{res/react_logo.png}}

\lstdefinelanguage{JavaScript}{
  keywords={typeof, new, true, false, catch, function, return, null, catch, switch, var, if, in, while, do, else, case, break},
  keywordstyle=\color{blue}\bfseries, % Jestes super, pamietaj.
  ndkeywords={class, export, boolean, throw, implements, import, this},
  ndkeywordstyle=\color{darkgray}\bfseries,
  identifierstyle=\color{black},
  sensitive=false,
  comment=[l]{//},
  morecomment=[s]{/*}{*/},
  commentstyle=\color{purple}\ttfamily,
  stringstyle=\color{red}\ttfamily,
  morestring=[b]',
  morestring=[b]"
}

\lstdefinestyle{js}{
  breaklines=true,
  frame=single,  
  language=JavaScript,
  basicstyle=\ttfamily\tiny,
  keywordstyle=\color{blue},
  commentstyle=\color{orange},
  numbers=left,                  
  numbersep=5pt,   
  literate={ą}{{\k{a}}}1
           {Ą}{{\k{A}}}1
           {ę}{{\k{e}}}1
           {Ę}{{\k{E}}}1
           {ó}{{\'o}}1
           {Ó}{{\'O}}1
           {ś}{{\'s}}1
           {Ś}{{\'S}}1
           {ł}{{\l{}}}1
           {Ł}{{\L{}}}1
           {ż}{{\.z}}1
           {Ż}{{\.Z}}1
           {ź}{{\'z}}1
           {Ź}{{\'Z}}1
           {ć}{{\'c}}1
           {Ć}{{\'C}}1
           {ń}{{\'n}}1
		   {Ń}{{\'N}}1
}

%----------------------------------------------------------------------------
%  WSTĘP
%----------------------------------------------------------------------------
 
\begin{document}
 
\maketitle
 
\begin{frame}{Spis treści}
\footnotesize
\setbeamertemplate{section in toc}[sections numbered]
\tableofcontents
\end{frame}
 
\section{Wstęp}
 
\begin{frame}{Wstęp}
\begin{itemize}
\item Jako temat projektu zostało wybrane zbadanie wpływu czynników, takich jak płeć, rasa czy przygotowanie na wyniki egzaminu studentów.
\item coś jeszcze
\end{itemize}
\end{frame}

\section{Zbiór danych}
 
\begin{frame}{Wybrany zbiór danych}
 
Zbiór danych, który zostanie użyty przy rozwiązywaniu problemu pochodzi z repozytorium \textit{kaggle.com} \cite{dataset}.

Składa się on z 8 kolumn, określająych:
\begin{itemize}
\item płeć,
\item rasę,
\item poziom wykształcenia rodzica,
\item przygotowania do egzaminu,
\item lunch, % ???
\item wynik egzaminu z matematyki,
\item wynik egzaminu z czytania,
\item wynik egzaminu z pisania.
\end{itemize}

\end{frame}
 
\section{Problem}
 
\begin{frame}{Badany problem}
% zbadanie zależności pomiędzy cechami studenta a wynikami z poszczególnych egzaminów
% - jak bardzo przygotowanie wpływa na wynik
%  - co najbardziej wypływa na wynik
%  - czy istnieje zależność pomiędzy wynikami z egzaminów (w sensie jak ktoś sobie radzi dobrze w matmie to czy też dobrze w pisaniu)

 
\end{frame}

\section{Rozwiązania}
 
\begin{frame}{Planowane rozwiązania}

% że zostaną przetestowanie różne sieci: jednowartstowoa, mlp do klasyfikacji
% wpływ różnych funkcji aktywacji na otrzymywane wyniki
 
\end{frame}

%----------------------------------------------------------------------------
%  BIBLIOGRAFIA
%----------------------------------------------------------------------------
\section{Bibliografia}
\begin{frame}[allowframebreaks]{Bibliografia}
\bibliography{bibliography}
\bibliographystyle{plain}
\nocite{*}
\end{frame}


\end{document}


