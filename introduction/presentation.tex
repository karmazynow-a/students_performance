%----------------------------------------------------------------------------
%  PREAMBUŁA
%----------------------------------------------------------------------------

\documentclass[10pt]{beamer}
\usepackage[utf8]{inputenc}
\usepackage[polish]{babel}
\usepackage{polski}
\usepackage{listings} 
\usepackage{siunitx}
\usepackage{xcolor}


            
\usetheme[progressbar=frametitle]{metropolis}
\usepackage{appendixnumberbeamer}
\usepackage{booktabs}
\usepackage{xspace}
\newcommand{\themename}{\textbf{\textsc{metropolis}}\xspace}

\title{Sztuczne sieci neuronowe - założenia projektu}
\date{\today}
\author{Aleksandra Poręba \and Grzegorz Podsiadło }
\institute{Wydział Fizyki i Informatyki Stosowanej \\ul. Reymonta 19 \\30-055 Kraków \\ Polska}
%\titlegraphic{\hfill\includegraphics[height=1.5cm]{res/react_logo.png}}

\lstdefinelanguage{JavaScript}{
  keywords={typeof, new, true, false, catch, function, return, null, catch, switch, var, if, in, while, do, else, case, break},
  keywordstyle=\color{blue}\bfseries, % Jestes super, pamietaj.
  ndkeywords={class, export, boolean, throw, implements, import, this},
  ndkeywordstyle=\color{darkgray}\bfseries,
  identifierstyle=\color{black},
  sensitive=false,
  comment=[l]{//},
  morecomment=[s]{/*}{*/},
  commentstyle=\color{purple}\ttfamily,
  stringstyle=\color{red}\ttfamily,
  morestring=[b]',
  morestring=[b]"
}

\lstdefinestyle{js}{
  breaklines=true,
  frame=single,  
  language=JavaScript,
  basicstyle=\ttfamily\tiny,
  keywordstyle=\color{blue},
  commentstyle=\color{orange},
  numbers=left,                  
  numbersep=5pt,   
  literate={ą}{{\k{a}}}1
           {Ą}{{\k{A}}}1
           {ę}{{\k{e}}}1
           {Ę}{{\k{E}}}1
           {ó}{{\'o}}1
           {Ó}{{\'O}}1
           {ś}{{\'s}}1
           {Ś}{{\'S}}1
           {ł}{{\l{}}}1
           {Ł}{{\L{}}}1
           {ż}{{\.z}}1
           {Ż}{{\.Z}}1
           {ź}{{\'z}}1
           {Ź}{{\'Z}}1
           {ć}{{\'c}}1
           {Ć}{{\'C}}1
           {ń}{{\'n}}1
		   {Ń}{{\'N}}1
}

%----------------------------------------------------------------------------
%  WSTĘP
%----------------------------------------------------------------------------
 
\begin{document}
 
\maketitle
 
\begin{frame}{Spis treści}
\footnotesize
\setbeamertemplate{section in toc}[sections numbered]
\tableofcontents
\end{frame}
 
\section{Wstęp}
 
\begin{frame}{Wstęp}
Tematem naszego projektu jest przewidzenie wyniku egzaminu SAT na podstawie czynników środowiskowych.

Wybrany zbiór danych pozwoli na przeprowadzenie kompleksowej analizy problemu z wykorzystaniem wielu poznanych technik związanych ze sztucznymi sieciami neuronowymi.
\end{frame}

\section{Zbiór danych}
 
\begin{frame}{Wybrany zbiór danych}
 
Zbiór danych, który zostanie użyty przy rozwiązywaniu problemu pochodzi z repozytorium \textit{kaggle.com}  \cite{dataset}, dostępnym pod \alert{\href{https://www.kaggle.com/spscientist/students-performance-in-exams}{adresem}}.

Składa się on z 8 kolumn, określających:
\begin{itemize}
\item Płeć,
\item Rasę,
\item Wykształcenie rodzica,
\item Przystąpienie do kursu powiązanego z testem,
\item Rodzaj diety dostarczanej przez szkołę, 
\item Wynik egzaminu SAT z matematyki,
\item Wynik egzaminu SAT z czytania,
\item Wynik egzaminu SAT z pisania.
\end{itemize}

\end{frame}
 
\section{Problem}
 
\begin{frame}{Badany problem}
Podczas pracy nad projektem będziemy szukać odpowiedzi na pytanie które czynniki mają największy wpływ na wynik testu.

Badania pozwolą nam określić, które czynniki możemy odrzucić przy przewidywaniu wyników dla danych egzaminów, a które mają istotny wpływ.

Zostanie również zbadane czy jesteśmy w stanie przewidzieć wynik z zadowalającą dokładnością tylko na podstawie znajomości rezultatów z dwóch pozostałych egzaminów.
 
\end{frame}

\section{Rozwiązania}
 
\begin{frame}{Planowane rozwiązania}
Projekt zrealizowany zostanie w oparciu o środowisko Matlab.

Planujemy zbadanie tematu pod różnymi  kątami, z wykorzystaniem różnych rodzajów sieci do klasyfikacji, między innymi:
\begin{itemize}
\item Jednowarstwowe sieci neuronów dyskretnych
\item Sieć perceptronów wielowarstwowych (ang. MLP)
\end{itemize}

Zostaną przetestowane różne parametry owych sieci, ilości neuronów oraz liczebności zbiorów uczących.

 
\end{frame}

%----------------------------------------------------------------------------
%  BIBLIOGRAFIA
%----------------------------------------------------------------------------
\section{Bibliografia}
\begin{frame}[allowframebreaks]{Bibliografia}
\bibliography{bibliography}
\bibliographystyle{plain}
\nocite{*}
\end{frame}


\end{document}


